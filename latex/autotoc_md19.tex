This is an Objective-\/C wrapper around S\+Q\+Lite\+: \href{http://sqlite.org/}{\tt http\+://sqlite.\+org/} .

O\+P\+T\+L\+Y\+F\+M\+DB is a modified copy of the original F\+M\+DB covered by the \char`\"{}\+License.\+txt\char`\"{} file mentioned at the end of this document.

\href{http://ccgus.github.io/fmdb/html/index.html}{\tt http\+://ccgus.\+github.\+io/fmdb/html/index.\+html}

You can use either style in your Cocoa project. O\+P\+T\+L\+Y\+F\+M\+DB will figure out which you are using at compile time and do the right thing.

O\+P\+T\+L\+Y\+F\+M\+DB 2.\+7 attempts to support a more natural interface. This represents a fairly significant change for Swift developers (audited for nullability; shifted to properties in external interfaces where possible rather than methods; etc.). For Objective-\/C developers, this should be a fairly seamless transition (unless you were using the ivars that were previously exposed in the public interface, which you shouldn\textquotesingle{}t have been doing, anyway!).

\subsubsection*{Nullability and Swift Optionals}

O\+P\+T\+L\+Y\+F\+M\+DB 2.\+7 is largely the same as prior versions, but has been audited for nullability. For Objective-\/C users, this simply means that if you perform a static analysis of your O\+P\+T\+L\+Y\+F\+M\+D\+B-\/based project, you may receive more meaningful warnings as you review your project, but there are likely to be few, if any, changes necessary in your code.

For Swift users, this nullability audit results in changes that are not entirely backward compatible with O\+P\+T\+L\+Y\+F\+M\+DB 2.\+6, but is a little more Swifty. Before O\+P\+T\+L\+Y\+F\+M\+DB was audited for nullability, Swift was forced to defensively assume that variables were optional, but the library now more accurately knows which properties and method parameters are optional, and which are not.

This means, though, that Swift code written for O\+P\+T\+L\+Y\+F\+M\+DB 2.\+7 may require changes. For example, consider the following Swift 3/\+Swift 4 code for O\+P\+T\+L\+Y\+F\+M\+DB 2.\+6\+: 
\begin{DoxyCode}
guard let queue = OPTLYFMDBDatabaseQueue(path: fileURL.path) else \{
    print("Unable to create OPTLYFMDBDatabaseQueue")
    return
\}

queue.inTransaction \{ db, rollback in
    do \{
        guard let db == db else \{
            // handle error here
            return
        \}

        try db.executeUpdate("INSERT INTO foo (bar) VALUES (?)", values: [1])
        try db.executeUpdate("INSERT INTO foo (bar) VALUES (?)", values: [2])
    \} catch \{
        rollback?.pointee = true
    \}
\}
\end{DoxyCode}


Because O\+P\+T\+L\+Y\+F\+M\+DB 2.\+6 was not audited for nullability, Swift inferred that {\ttfamily db} and {\ttfamily rollback} were optionals. But, now, in O\+P\+T\+L\+Y\+F\+M\+DB 2.\+7, Swift now knows that, for example, neither {\ttfamily db} nor {\ttfamily rollback} above can be {\ttfamily nil}, so they are no longer optionals. Thus it becomes\+:


\begin{DoxyCode}
let queue = OPTLYFMDBDatabaseQueue(url: fileURL)

queue.inTransaction \{ db, rollback in
    do \{
        try db.executeUpdate("INSERT INTO foo (bar) VALUES (?)", values: [1])
        try db.executeUpdate("INSERT INTO foo (bar) VALUES (?)", values: [2])
    \} catch \{
        rollback.pointee = true
    \}
\}
\end{DoxyCode}


\subsubsection*{Custom Functions}

In the past, when writing custom functions, you would have to generally include your own {\ttfamily @autoreleasepool} block to avoid problems when writing functions that scanned through a large table. Now, O\+P\+T\+L\+Y\+F\+M\+DB will automatically wrap it in an autorelease pool, so you don\textquotesingle{}t have to.

Also, in the past, when retrieving the values passed to the function, you had to drop down to the S\+Q\+Lite C A\+PI and include your own {\ttfamily sqlite3\+\_\+value\+\_\+\+X\+XX} calls. There are now {\ttfamily \mbox{\hyperlink{interface_o_p_t_l_y_f_m_d_b_database}{O\+P\+T\+L\+Y\+F\+M\+D\+B\+Database}}} methods, {\ttfamily value\+Int}, {\ttfamily value\+String}, etc., so you can stay within Swift and/or Objective-\/C, without needing to call the C functions yourself. Likewise, when specifying the return values, you no longer need to call {\ttfamily sqlite3\+\_\+result\+\_\+\+X\+XX} C A\+PI, but rather you can use {\ttfamily \mbox{\hyperlink{interface_o_p_t_l_y_f_m_d_b_database}{O\+P\+T\+L\+Y\+F\+M\+D\+B\+Database}}} methods, {\ttfamily result\+Int}, {\ttfamily result\+String}, etc. There is a new {\ttfamily enum} for {\ttfamily value\+Type} called {\ttfamily Sqlite\+Value\+Type}, which can be used for checking the type of parameter passed to the custom function.

Thus, you can do something like (as of Swift 3)\+:


\begin{DoxyCode}
db.makeFunctionNamed("RemoveDiacritics", arguments: 1) \{ context, argc, argv in
    guard db.valueType(argv[0]) == .text || db.valueType(argv[0]) == .null else \{
        db.resultError("Expected string parameter", context: context)
        return
    \}

    if let string = db.valueString(argv[0])?.folding(options: .diacriticInsensitive, locale: nil) \{
        db.resultString(string, context: context)
    \} else \{
        db.resultNull(context: context)
    \}
\}
\end{DoxyCode}


And you can then use that function in your S\+QL (in this case, matching both \char`\"{}\+Jose\char`\"{} and \char`\"{}\+José\char`\"{})\+:


\begin{DoxyCode}
\textcolor{keyword}{SELECT} * \textcolor{keyword}{FROM} employees \textcolor{keyword}{WHERE} RemoveDiacritics(first\_name) \textcolor{keyword}{LIKE} \textcolor{stringliteral}{'jose'}
\end{DoxyCode}


Note, the method {\ttfamily make\+Function\+Named\+:maximum\+Arguments\+:with\+Block\+:} has been renamed to {\ttfamily make\+Function\+Named\+:arguments\+:block\+:}, to more accurately reflect the functional intent of the second parameter.

\subsubsection*{A\+PI Changes}

In addition to the {\ttfamily make\+Function\+Named} noted above, there are a few other A\+PI changes. Specifically,


\begin{DoxyItemize}
\item To become consistent with the rest of the A\+PI, the methods {\ttfamily object\+For\+Column\+Name} and {\ttfamily U\+T\+F8\+String\+For\+Column\+Name} have been renamed to {\ttfamily object\+For\+Column} and {\ttfamily U\+T\+F8\+String\+For\+Column}.
\item Note, the {\ttfamily object\+For\+Column} (and the associted subscript operator) now returns {\ttfamily nil} if an invalid column name/index is passed to it. It used to return {\ttfamily N\+S\+Null}.
\item To avoid confusion with {\ttfamily \mbox{\hyperlink{interface_o_p_t_l_y_f_m_d_b_database_queue}{O\+P\+T\+L\+Y\+F\+M\+D\+B\+Database\+Queue}}} method {\ttfamily in\+Transaction}, which performs transactions, the {\ttfamily \mbox{\hyperlink{interface_o_p_t_l_y_f_m_d_b_database}{O\+P\+T\+L\+Y\+F\+M\+D\+B\+Database}}} method to determine whether you are in a transaction or not, {\ttfamily in\+Transaction}, has been replaced with a read-\/only property, {\ttfamily is\+In\+Transaction}.
\item Several functions have been converted to properties, namely, {\ttfamily database\+Path}, {\ttfamily max\+Busy\+Retry\+Time\+Interval}, {\ttfamily should\+Cache\+Statements}, {\ttfamily sqlite\+Handle}, {\ttfamily has\+Open\+Result\+Sets}, {\ttfamily last\+Insert\+Row\+Id}, {\ttfamily changes}, {\ttfamily good\+Connection}, {\ttfamily column\+Count}, {\ttfamily result\+Dictionary}, {\ttfamily application\+ID}, {\ttfamily application\+I\+D\+String}, {\ttfamily user\+Version}, {\ttfamily count\+Of\+Checked\+In\+Databases}, {\ttfamily count\+Of\+Checked\+Out\+Databases}, and {\ttfamily count\+Of\+Open\+Databases}. For Objective-\/C users, this has little material impact, but for Swift users, it results in a slightly more natural interface. Note\+: For Objective-\/C developers, previously versions of O\+P\+T\+L\+Y\+F\+M\+DB exposed many ivars (but we hope you weren\textquotesingle{}t using them directly, anyway!), but the implmentation details for these are no longer exposed.
\end{DoxyItemize}

\subsubsection*{U\+RL Methods}

In keeping with Apple\textquotesingle{}s shift from paths to U\+R\+Ls, there are now {\ttfamily N\+S\+U\+RL} renditions of the various {\ttfamily init} methods, previously only accepting paths.

There are three main classes in O\+P\+T\+L\+Y\+F\+M\+DB\+:


\begin{DoxyEnumerate}
\item {\ttfamily \mbox{\hyperlink{interface_o_p_t_l_y_f_m_d_b_database}{O\+P\+T\+L\+Y\+F\+M\+D\+B\+Database}}} -\/ Represents a single S\+Q\+Lite database. Used for executing S\+QL statements.
\item {\ttfamily \mbox{\hyperlink{interface_o_p_t_l_y_f_m_d_b_result_set}{O\+P\+T\+L\+Y\+F\+M\+D\+B\+Result\+Set}}} -\/ Represents the results of executing a query on an {\ttfamily \mbox{\hyperlink{interface_o_p_t_l_y_f_m_d_b_database}{O\+P\+T\+L\+Y\+F\+M\+D\+B\+Database}}}.
\item {\ttfamily \mbox{\hyperlink{interface_o_p_t_l_y_f_m_d_b_database_queue}{O\+P\+T\+L\+Y\+F\+M\+D\+B\+Database\+Queue}}} -\/ If you\textquotesingle{}re wanting to perform queries and updates on multiple threads, you\textquotesingle{}ll want to use this class. It\textquotesingle{}s described in the \char`\"{}\+Thread Safety\char`\"{} section below.
\end{DoxyEnumerate}

\subsubsection*{Database Creation}

An {\ttfamily \mbox{\hyperlink{interface_o_p_t_l_y_f_m_d_b_database}{O\+P\+T\+L\+Y\+F\+M\+D\+B\+Database}}} is created with a path to a S\+Q\+Lite database file. This path can be one of these three\+:


\begin{DoxyEnumerate}
\item A file system path. The file does not have to exist on disk. If it does not exist, it is created for you.
\item An empty string ({\ttfamily @\char`\"{}\char`\"{}}). An empty database is created at a temporary location. This database is deleted with the {\ttfamily \mbox{\hyperlink{interface_o_p_t_l_y_f_m_d_b_database}{O\+P\+T\+L\+Y\+F\+M\+D\+B\+Database}}} connection is closed.
\item {\ttfamily N\+U\+LL}. An in-\/memory database is created. This database will be destroyed with the {\ttfamily \mbox{\hyperlink{interface_o_p_t_l_y_f_m_d_b_database}{O\+P\+T\+L\+Y\+F\+M\+D\+B\+Database}}} connection is closed.
\end{DoxyEnumerate}

(For more information on temporary and in-\/memory databases, read the sqlite documentation on the subject\+: \href{http://www.sqlite.org/inmemorydb.html}{\tt http\+://www.\+sqlite.\+org/inmemorydb.\+html})


\begin{DoxyCode}
NSString *path = [NSTemporaryDirectory() stringByAppendingPathComponent:@"tmp.db"];
OPTLYFMDBDatabase *db = [OPTLYFMDBDatabase databaseWithPath:path];
\end{DoxyCode}


\subsubsection*{Opening}

Before you can interact with the database, it must be opened. Opening fails if there are insufficient resources or permissions to open and/or create the database.


\begin{DoxyCode}
if (![db open]) \{
    // [db release];   // uncomment this line in manual referencing code; in ARC, this is not
       necessary/permitted
    db = nil;
    return;
\}
\end{DoxyCode}


\subsubsection*{Executing Updates}

Any sort of S\+QL statement which is not a {\ttfamily S\+E\+L\+E\+CT} statement qualifies as an update. This includes {\ttfamily C\+R\+E\+A\+TE}, {\ttfamily U\+P\+D\+A\+TE}, {\ttfamily I\+N\+S\+E\+RT}, {\ttfamily A\+L\+T\+ER}, {\ttfamily C\+O\+M\+M\+IT}, {\ttfamily B\+E\+G\+IN}, {\ttfamily D\+E\+T\+A\+CH}, {\ttfamily D\+E\+L\+E\+TE}, {\ttfamily D\+R\+OP}, {\ttfamily E\+ND}, {\ttfamily E\+X\+P\+L\+A\+IN}, {\ttfamily V\+A\+C\+U\+UM}, and {\ttfamily R\+E\+P\+L\+A\+CE} statements (plus many more). Basically, if your S\+QL statement does not begin with {\ttfamily S\+E\+L\+E\+CT}, it is an update statement.

Executing updates returns a single value, a {\ttfamily B\+O\+OL}. A return value of {\ttfamily Y\+ES} means the update was successfully executed, and a return value of {\ttfamily NO} means that some error was encountered. You may invoke the {\ttfamily -\/last\+Error\+Message} and {\ttfamily -\/last\+Error\+Code} methods to retrieve more information.

\subsubsection*{Executing Queries}

A {\ttfamily S\+E\+L\+E\+CT} statement is a query and is executed via one of the {\ttfamily -\/execute\+Query...} methods.

Executing queries returns an {\ttfamily \mbox{\hyperlink{interface_o_p_t_l_y_f_m_d_b_result_set}{O\+P\+T\+L\+Y\+F\+M\+D\+B\+Result\+Set}}} object if successful, and {\ttfamily nil} upon failure. You should use the {\ttfamily -\/last\+Error\+Message} and {\ttfamily -\/last\+Error\+Code} methods to determine why a query failed.

In order to iterate through the results of your query, you use a {\ttfamily while()} loop. You also need to \char`\"{}step\char`\"{} from one record to the other. With O\+P\+T\+L\+Y\+F\+M\+DB, the easiest way to do that is like this\+:


\begin{DoxyCode}
OPTLYFMDBResultSet *s = [db executeQuery:@"SELECT * FROM myTable"];
while ([s next]) \{
    //retrieve values for each record
\}
\end{DoxyCode}


You must always invoke {\ttfamily -\/\mbox{[}\mbox{\hyperlink{interface_o_p_t_l_y_f_m_d_b_result_set}{O\+P\+T\+L\+Y\+F\+M\+D\+B\+Result\+Set}} next\mbox{]}} before attempting to access the values returned in a query, even if you\textquotesingle{}re only expecting one\+:


\begin{DoxyCode}
OPTLYFMDBResultSet *s = [db executeQuery:@"SELECT COUNT(*) FROM myTable"];
if ([s next]) \{
    int totalCount = [s intForColumnIndex:0];
\}
\end{DoxyCode}


{\ttfamily \mbox{\hyperlink{interface_o_p_t_l_y_f_m_d_b_result_set}{O\+P\+T\+L\+Y\+F\+M\+D\+B\+Result\+Set}}} has many methods to retrieve data in an appropriate format\+:


\begin{DoxyItemize}
\item {\ttfamily int\+For\+Column\+:}
\item {\ttfamily long\+For\+Column\+:}
\item {\ttfamily long\+Long\+Int\+For\+Column\+:}
\item {\ttfamily bool\+For\+Column\+:}
\item {\ttfamily double\+For\+Column\+:}
\item {\ttfamily string\+For\+Column\+:}
\item {\ttfamily date\+For\+Column\+:}
\item {\ttfamily data\+For\+Column\+:}
\item {\ttfamily data\+No\+Copy\+For\+Column\+:}
\item {\ttfamily U\+T\+F8\+String\+For\+Column\+:}
\item {\ttfamily object\+For\+Column\+:}
\end{DoxyItemize}

Each of these methods also has a {\ttfamily \{type\}For\+Column\+Index\+:} variant that is used to retrieve the data based on the position of the column in the results, as opposed to the column\textquotesingle{}s name.

Typically, there\textquotesingle{}s no need to {\ttfamily -\/close} an {\ttfamily \mbox{\hyperlink{interface_o_p_t_l_y_f_m_d_b_result_set}{O\+P\+T\+L\+Y\+F\+M\+D\+B\+Result\+Set}}} yourself, since that happens when either the result set is deallocated, or the parent database is closed.

\subsubsection*{Closing}

When you have finished executing queries and updates on the database, you should {\ttfamily -\/close} the {\ttfamily \mbox{\hyperlink{interface_o_p_t_l_y_f_m_d_b_database}{O\+P\+T\+L\+Y\+F\+M\+D\+B\+Database}}} connection so that S\+Q\+Lite will relinquish any resources it has acquired during the course of its operation.


\begin{DoxyCode}
[db close];
\end{DoxyCode}


\subsubsection*{Transactions}

{\ttfamily \mbox{\hyperlink{interface_o_p_t_l_y_f_m_d_b_database}{O\+P\+T\+L\+Y\+F\+M\+D\+B\+Database}}} can begin and commit a transaction by invoking one of the appropriate methods or executing a begin/end transaction statement.

\subsubsection*{Multiple Statements and Batch Stuff}

You can use {\ttfamily \mbox{\hyperlink{interface_o_p_t_l_y_f_m_d_b_database}{O\+P\+T\+L\+Y\+F\+M\+D\+B\+Database}}}\textquotesingle{}s execute\+Statements\+:with\+Result\+Block\+: to do multiple statements in a string\+:


\begin{DoxyCode}
NSString *sql = @"create table bulktest1 (id integer primary key autoincrement, x text);"
                 "create table bulktest2 (id integer primary key autoincrement, y text);"
                 "create table bulktest3 (id integer primary key autoincrement, z text);"
                 "insert into bulktest1 (x) values ('XXX');"
                 "insert into bulktest2 (y) values ('YYY');"
                 "insert into bulktest3 (z) values ('ZZZ');";

success = [db executeStatements:sql];

sql = @"select count(*) as count from bulktest1;"
       "select count(*) as count from bulktest2;"
       "select count(*) as count from bulktest3;";

success = [self.db executeStatements:sql withResultBlock:^int(NSDictionary *dictionary) \{
    NSInteger count = [dictionary[@"count"] integerValue];
    XCTAssertEqual(count, 1, @"expected one record for dictionary %@", dictionary);
    return 0;
\}];
\end{DoxyCode}


\subsubsection*{Data Sanitization}

When providing a S\+QL statement to O\+P\+T\+L\+Y\+F\+M\+DB, you should not attempt to \char`\"{}sanitize\char`\"{} any values before insertion. Instead, you should use the standard S\+Q\+Lite binding syntax\+:


\begin{DoxyCode}
\textcolor{keyword}{INSERT} \textcolor{keyword}{INTO} myTable \textcolor{keyword}{VALUES} (?, ?, ?, ?)
\end{DoxyCode}


The {\ttfamily ?} character is recognized by S\+Q\+Lite as a placeholder for a value to be inserted. The execution methods all accept a variable number of arguments (or a representation of those arguments, such as an {\ttfamily N\+S\+Array}, {\ttfamily N\+S\+Dictionary}, or a {\ttfamily va\+\_\+list}), which are properly escaped for you.

And, to use that S\+QL with the {\ttfamily ?} placeholders from Objective-\/C\+:


\begin{DoxyCode}
NSInteger identifier = 42;
NSString *name = @"Liam O'Flaherty (\(\backslash\)"the famous Irish author\(\backslash\)")";
NSDate *date = [NSDate date];
NSString *comment = nil;

BOOL success = [db executeUpdate:@"INSERT INTO authors (identifier, name, date, comment) VALUES (?, ?, ?,
       ?)", @(identifier), name, date, comment ?: [NSNull null]];
if (!success) \{
    NSLog(@"error = %@", [db lastErrorMessage]);
\}
\end{DoxyCode}


\begin{quote}
{\bfseries Note\+:} Fundamental data types, like the {\ttfamily N\+S\+Integer} variable {\ttfamily identifier}, should be as a {\ttfamily N\+S\+Number} objects, achieved by using the {\ttfamily @} syntax, shown above. Or you can use the {\ttfamily \mbox{[}N\+S\+Number number\+With\+Int\+:identifier\mbox{]}} syntax, too.

Likewise, S\+QL {\ttfamily N\+U\+LL} values should be inserted as {\ttfamily \mbox{[}N\+S\+Null null\mbox{]}}. For example, in the case of {\ttfamily comment} which might be {\ttfamily nil} (and is in this example), you can use the {\ttfamily comment ?\+: \mbox{[}N\+S\+Null null\mbox{]}} syntax, which will insert the string if {\ttfamily comment} is not {\ttfamily nil}, but will insert {\ttfamily \mbox{[}N\+S\+Null null\mbox{]}} if it is {\ttfamily nil}. \end{quote}


In Swift, you would use {\ttfamily execute\+Update(values\+:)}, which not only is a concise Swift syntax, but also {\ttfamily throws} errors for proper error handling\+:


\begin{DoxyCode}
do \{
    let identifier = 42
    let name = "Liam O'Flaherty (\(\backslash\)"the famous Irish author\(\backslash\)")"
    let date = Date()
    let comment: String? = nil

    try db.executeUpdate("INSERT INTO authors (identifier, name, date, comment) VALUES (?, ?, ?, ?)",
       values: [identifier, name, date, comment ?? NSNull()])
\} catch \{
    print("error = \(\backslash\)(error)")
\}
\end{DoxyCode}


\begin{quote}
{\bfseries Note\+:} In Swift, you don\textquotesingle{}t have to wrap fundamental numeric types like you do in Objective-\/C. But if you are going to insert an optional string, you would probably use the {\ttfamily comment ?? N\+S\+Null()} syntax (i.\+e., if it is {\ttfamily nil}, use {\ttfamily N\+S\+Null}, otherwise use the string). \end{quote}


Alternatively, you may use named parameters syntax\+:


\begin{DoxyCode}
\textcolor{keyword}{INSERT} \textcolor{keyword}{INTO} authors (identifier, name, \textcolor{keywordtype}{date}, comment) \textcolor{keyword}{VALUES} (:identifier, :name, :\textcolor{keywordtype}{date}, :comment)
\end{DoxyCode}


The parameters {\itshape must} start with a colon. S\+Q\+Lite itself supports other characters, but internally the dictionary keys are prefixed with a colon, do {\bfseries not} include the colon in your dictionary keys.


\begin{DoxyCode}
NSDictionary *arguments =@"identifier": @(identifier), @"name": name, @"date": date, @"comment": comment ?:
       [NSNull null]\};
BOOL success = [db executeUpdate:@"INSERT INTO authors (identifier, name, date, comment) VALUES
       (:identifier, :name, :date, :comment)" withParameterDictionary:arguments];
if (!success) \{
    NSLog(@"error = %@", [db lastErrorMessage]);
\}
\end{DoxyCode}


The key point is that one should not use {\ttfamily N\+S\+String} method {\ttfamily string\+With\+Format} to manually insert values into the S\+QL statement, itself. Nor should one Swift string interpolation to insert values into the S\+QL. Use {\ttfamily ?} placeholders for values to be inserted into the database (or used in {\ttfamily W\+H\+E\+RE} clauses in {\ttfamily S\+E\+L\+E\+CT} statements).

\subsection*{Using \mbox{\hyperlink{interface_o_p_t_l_y_f_m_d_b_database_queue}{O\+P\+T\+L\+Y\+F\+M\+D\+B\+Database\+Queue}} and Thread Safety.}

Using a single instance of {\ttfamily \mbox{\hyperlink{interface_o_p_t_l_y_f_m_d_b_database}{O\+P\+T\+L\+Y\+F\+M\+D\+B\+Database}}} from multiple threads at once is a bad idea. It has always been OK to make a {\ttfamily \mbox{\hyperlink{interface_o_p_t_l_y_f_m_d_b_database}{O\+P\+T\+L\+Y\+F\+M\+D\+B\+Database}}} object {\itshape per thread}. Just don\textquotesingle{}t share a single instance across threads, and definitely not across multiple threads at the same time. Bad things will eventually happen and you\textquotesingle{}ll eventually get something to crash, or maybe get an exception, or maybe meteorites will fall out of the sky and hit your Mac Pro. {\itshape This would suck}.

{\bfseries So don\textquotesingle{}t instantiate a single {\ttfamily \mbox{\hyperlink{interface_o_p_t_l_y_f_m_d_b_database}{O\+P\+T\+L\+Y\+F\+M\+D\+B\+Database}}} object and use it across multiple threads.}

Instead, use {\ttfamily \mbox{\hyperlink{interface_o_p_t_l_y_f_m_d_b_database_queue}{O\+P\+T\+L\+Y\+F\+M\+D\+B\+Database\+Queue}}}. Instantiate a single {\ttfamily \mbox{\hyperlink{interface_o_p_t_l_y_f_m_d_b_database_queue}{O\+P\+T\+L\+Y\+F\+M\+D\+B\+Database\+Queue}}} and use it across multiple threads. The {\ttfamily \mbox{\hyperlink{interface_o_p_t_l_y_f_m_d_b_database_queue}{O\+P\+T\+L\+Y\+F\+M\+D\+B\+Database\+Queue}}} object will synchronize and coordinate access across the multiple threads. Here\textquotesingle{}s how to use it\+:

First, make your queue.


\begin{DoxyCode}
OPTLYFMDBDatabaseQueue *queue = [OPTLYFMDBDatabaseQueue databaseQueueWithPath:aPath];
\end{DoxyCode}


Then use it like so\+:


\begin{DoxyCode}
[queue inDatabase:^(OPTLYFMDBDatabase *db) \{
    [db executeUpdate:@"INSERT INTO myTable VALUES (?)", @1];
    [db executeUpdate:@"INSERT INTO myTable VALUES (?)", @2];
    [db executeUpdate:@"INSERT INTO myTable VALUES (?)", @3];

    OPTLYFMDBResultSet *rs = [db executeQuery:@"select * from foo"];
    while ([rs next]) \{
        …
    \}
\}];
\end{DoxyCode}


An easy way to wrap things up in a transaction can be done like this\+:


\begin{DoxyCode}
[queue inTransaction:^(OPTLYFMDBDatabase *db, BOOL *rollback) \{
    [db executeUpdate:@"INSERT INTO myTable VALUES (?)", @1];
    [db executeUpdate:@"INSERT INTO myTable VALUES (?)", @2];
    [db executeUpdate:@"INSERT INTO myTable VALUES (?)", @3];

    if (whoopsSomethingWrongHappened) \{
        *rollback = YES;
        return;
    \}

    // etc ...
\}];
\end{DoxyCode}


The Swift equivalent would be\+:


\begin{DoxyCode}
queue.inTransaction \{ db, rollback in
    do \{
        try db.executeUpdate("INSERT INTO myTable VALUES (?)", values: [1])
        try db.executeUpdate("INSERT INTO myTable VALUES (?)", values: [2])
        try db.executeUpdate("INSERT INTO myTable VALUES (?)", values: [3])

        if whoopsSomethingWrongHappened \{
            rollback.pointee = true
            return
        \}

        // etc ...
    \} catch \{
        rollback.pointee = true
        print(error)
    \}
\}
\end{DoxyCode}


(Note, as of Swift 3, use {\ttfamily pointee}. But in Swift 2.\+3, use {\ttfamily memory} rather than {\ttfamily pointee}.)

{\ttfamily \mbox{\hyperlink{interface_o_p_t_l_y_f_m_d_b_database_queue}{O\+P\+T\+L\+Y\+F\+M\+D\+B\+Database\+Queue}}} will run the blocks on a serialized queue (hence the name of the class). So if you call {\ttfamily \mbox{\hyperlink{interface_o_p_t_l_y_f_m_d_b_database_queue}{O\+P\+T\+L\+Y\+F\+M\+D\+B\+Database\+Queue}}}\textquotesingle{}s methods from multiple threads at the same time, they will be executed in the order they are received. This way queries and updates won\textquotesingle{}t step on each other\textquotesingle{}s toes, and every one is happy.

{\bfseries Note\+:} The calls to {\ttfamily \mbox{\hyperlink{interface_o_p_t_l_y_f_m_d_b_database_queue}{O\+P\+T\+L\+Y\+F\+M\+D\+B\+Database\+Queue}}}\textquotesingle{}s methods are blocking. So even though you are passing along blocks, they will {\bfseries not} be run on another thread.

You can do this! For an example, look for {\ttfamily -\/make\+Function\+Named\+:} in main.\+m

You can use O\+P\+T\+L\+Y\+F\+M\+DB in Swift projects too.

To do this, you must\+:


\begin{DoxyEnumerate}
\item Copy the relevant {\ttfamily .m} and {\ttfamily .h} files from the O\+P\+T\+L\+Y\+F\+M\+DB {\ttfamily src} folder into your project.

You can copy all of them (which is easiest), or only the ones you need. Likely you will need \href{http://ccgus.github.io/optlyfmdb/html/Classes/OPTLYFMDBDatabase.html}{\tt {\ttfamily O\+P\+T\+L\+Y\+F\+M\+D\+B\+Database}} and \href{http://ccgus.github.io/optlyfmdb/html/Classes/OPTLYFMDBResultSet.html}{\tt {\ttfamily O\+P\+T\+L\+Y\+F\+M\+D\+B\+Result\+Set}} at a minimum. If you are doing multithreaded access to a database, \href{http://ccgus.github.io/optlyfmdb/html/Classes/OPTLYFMDBDatabaseQueue.html}{\tt {\ttfamily O\+P\+T\+L\+Y\+F\+M\+D\+B\+Database\+Queue}} is quite useful, too. If you choose to not copy all of the files from the {\ttfamily src} directory, though, you may want to update {\ttfamily \mbox{\hyperlink{_o_p_t_l_y_f_m_d_b_8h_source}{O\+P\+T\+L\+Y\+F\+M\+D\+B.\+h}}} to only reference the files that you included in your project.

Note, if you\textquotesingle{}re copying all of the files from the {\ttfamily src} folder into to your project (which is recommended), you may want to drag the individual files into your project, not the folder, itself, because if you drag the folder, you won\textquotesingle{}t be prompted to add the bridging header (see next point).
\item If prompted to create a \char`\"{}bridging header\char`\"{}, you should do so. If not prompted and if you don\textquotesingle{}t already have a bridging header, add one.

For more information on bridging headers, see \href{https://developer.apple.com/library/ios/documentation/Swift/Conceptual/BuildingCocoaApps/MixandMatch.html#//apple_ref/doc/uid/TP40014216-CH10-XID_76}{\tt Swift and Objective-\/C in the Same Project}.
\item In your bridging header, add a line that says\+: \`{}\`{}\`{}objc \#import \char`\"{}\+O\+P\+T\+L\+Y\+F\+M\+D\+B.\+h\char`\"{} \`{}\`{}\`{}
\item Use the variations of {\ttfamily execute\+Query} and {\ttfamily execute\+Update} with the {\ttfamily sql} and {\ttfamily values} parameters with {\ttfamily try} pattern, as shown below. These renditions of {\ttfamily execute\+Query} and {\ttfamily execute\+Update} both {\ttfamily throw} errors in true Swift fashion.
\end{DoxyEnumerate}

If you do the above, you can then write Swift code that uses {\ttfamily \mbox{\hyperlink{interface_o_p_t_l_y_f_m_d_b_database}{O\+P\+T\+L\+Y\+F\+M\+D\+B\+Database}}}. For example, as of Swift 3\+:


\begin{DoxyCode}
let fileURL = try! FileManager.default
    .url(for: .documentDirectory, in: .userDomainMask, appropriateFor: nil, create: false)
    .appendingPathComponent("test.sqlite")

let database = OPTLYFMDBDatabase(url: fileURL)

guard database.open() else \{
    print("Unable to open database")
    return
\}

do \{
    try database.executeUpdate("create table test(x text, y text, z text)", values: nil)
    try database.executeUpdate("insert into test (x, y, z) values (?, ?, ?)", values: ["a", "b", "c"])
    try database.executeUpdate("insert into test (x, y, z) values (?, ?, ?)", values: ["e", "f", "g"])

    let rs = try database.executeQuery("select x, y, z from test", values: nil)
    while rs.next() \{
        if let x = rs.string(forColumn: "x"), let y = rs.string(forColumn: "y"), let z =
       rs.string(forColumn: "z") \{
            print("x = \(\backslash\)(x); y = \(\backslash\)(y); z = \(\backslash\)(z)")
        \}
    \}
\} catch \{
    print("failed: \(\backslash\)(error.localizedDescription)")
\}

database.close()
\end{DoxyCode}


The license for F\+M\+DB is contained in the \char`\"{}\+License.\+txt\char`\"{} file.

If you happen to come across either Gus Mueller or Rob Ryan in a bar, you might consider purchasing a drink of their choosing if F\+M\+DB has been useful to you.

(The drink is for them of course, shame on you for trying to keep it.)

Copyright 2017, \mbox{\hyperlink{interface_optimizely}{Optimizely}}, Inc. and contributors

Licensed under the Apache License, Version 2.\+0 (the \char`\"{}\+License\char`\"{}); you may not use this file except in compliance with the License. You may obtain a copy of the License at

\href{http://www.apache.org/licenses/LICENSE-2.0}{\tt http\+://www.\+apache.\+org/licenses/\+L\+I\+C\+E\+N\+S\+E-\/2.\+0}

Unless required by applicable law or agreed to in writing, software distributed under the License is distributed on an \char`\"{}\+A\+S I\+S\char`\"{} B\+A\+S\+IS, W\+I\+T\+H\+O\+UT W\+A\+R\+R\+A\+N\+T\+I\+ES OR C\+O\+N\+D\+I\+T\+I\+O\+NS OF A\+NY K\+I\+ND, either express or implied. See the License for the specific language governing permissions and limitations under the License. 